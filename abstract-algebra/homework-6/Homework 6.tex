\documentclass[12pt]{article}
\usepackage[margin=1in]{geometry} 
\usepackage{amsmath,amsthm,amssymb,amsfonts,stmaryrd}
\usepackage{enumitem}
\usepackage{tabu}
\usepackage{fixltx2e}
\usepackage{xcolor}
 
\newcommand{\N}{\mathbb{N}}
\newcommand{\Z}{\mathbb{Z}}
 
\newenvironment{problem}[2][Problem]{\begin{trivlist}
\item[\hskip \labelsep {\bfseries #1}\hskip \labelsep {\bfseries #2.}]}{\end{trivlist}}
 
\begin{document}
\title{Math 310 Homework 6}
\author{Jacob Shiohira}
\maketitle

\noindent
\textit{Note:} This homework took a total of 6 hours. I initially did it alone, but I did review with Jacob Warner.

\begin{problem}{1} Section 2.2 \#13 \\

\noindent
\textbf{Proposition}: Prove or disprove: If $[a] \varodot [b] = [a] \varodot [c]$ and $[a] \neq [0]$ in $\Z_n$, then $[b] = [c]$.
\end{problem}

\begin{problem}{2} Section 2.3 \#1,2 \\

\noindent
\textbf{Proposition}: Find all the units and zero divisors in (Do both together): \\

\begin{enumerate} [label=(\alph*)]
\item $\Z_7$ Hello 
\item $\Z_8$ Hello 
\item $\Z_9$ Hello 
\item $\Z_10$
\end{enumerate}

\end{problem}

\begin{problem}{3} Section 2.3 \#8 \\

\noindent
\textbf{Proposition}: 
\begin{enumerate}[label=(\alph*)]
\item Give three examples of equations of the form $ax=b$ in $\Z_12$ that have no nonzero solutions.
\item For each of the equations in part $(a)$, does the equation $ax=0$ have a nonzero solution? 
\end{enumerate}
\end{problem}

\begin{problem}{4} Section 2.3 \#11 \\

\noindent
\textbf{Proposition}: Without using Exercises $13$ and $14$, prove: If $a,b \in \Z_n$ and $a$ is a unit, then the equation $ax=b$ has a unique solution in $\Z_n$. [\textit{Note:} You must find a solution for the equation \textit{and} show that this solution is the only one.]
\end{problem}

\begin{problem}{5} Section 2.3 \#13 \\

\noindent
\textbf{Proposition}: Let $a,b,n$ be integers with $n>1$. Let $d=(a,n)$ and assume $d|b$. Prove that the equation $[a]x=[b]$ has a solution in $Z_n$ as follows 
\begin{enumerate}[label=(\alph*)]
\item Explain why there are integers $u,v,a_1,b_1,n_1$ such that $au_nv=d$,$a=da_1$, $b=db_1$, and $n=dn_1$.
\item Show that each of 
\begin{center}
$[ub_1]$,$[ub_1+n_1]$,$[ub_1+2n_1]$,$[ub_1+3n_1]$,$\cdot \cdot \cdot$,$[ub_1+(d-1)n_1]$
\end{center}
is a solution of $[a]x=[b]$. \\
\end{enumerate}
\end{problem}

\begin{problem}{6} Section 2.3 \#14 \\

\noindent
\textbf{Proposition}: Let $a,b,n$ be integers with $n>1$. Let $d=(a,n)$ and assume $d|b$. Prove that the equation $[a]x=[b]$ has $d$ unique solutions in $Z_n$ as follows 
\begin{enumerate}[label=(\alph*)]
\item Show that the solutions listed in Exercise $13(b)$ are all distinct. [\textit{Hint}: $[r] = [s]$ if and only if $n|(r-s)$.]
\item If $x=[r]$ is any solution of $[a]x=[b]$, show that $[r]=[ub_1+kn_1]$ for some integer $k$ with $0 \leq k \leq d - 1$. [\textit{Hint}: $[ar]-[aub_1]=[0]$ (Why?), so that $n|(a(r-ub_1))$. Show that $n_1|(a_1(r-ub_1))$ and use Theorem $1.4$ to show that $n_1|(r-ub_1)$.]
\end{enumerate}
\end{problem}

\begin{problem}{7} Section 2.3 \#17 \\

\noindent
\textbf{Proposition}: Prove that the product of two units in $\Z_n$ is also a unit.
\end{problem}
\end{document} 