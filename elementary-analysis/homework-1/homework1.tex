\documentclass{article}

%----------------------------------------------------------------------------------------

\usepackage{listings} % Required for inserting code snippets
\usepackage[usenames,dvipsnames]{color} % Required for specifying custom colors and referring to colors by name
\usepackage{amssymb}
\usepackage{amsmath}
\usepackage{mathtools}
\usepackage{tikz}

\definecolor{DarkGreen}{rgb}{0.0,0.4,0.0} % Comment color
\definecolor{highlight}{RGB}{255,251,204} % Code highlight color

%----------------------------------------------------------------------------------------

\begin{document}

%----------------------------------------------------------------------------------------

\begin{flushleft}
Suppose that $A, B, C$ are sets satisfying \\
\begin{center}
(1) $A \cap B = A \cap C$  and (2) $A \cup B = A \cup C$
\end{center}
Prove that $B=C$
\end{flushleft}

\begin{flushleft}
\textbf{PROOF}
\end{flushleft}

% STATEMENT 1
\begin{flushleft}
 (1)\:\: For every $x \in A \cap B$, $x \in A$ and $x \in B$ simultaneously. Since \\
\qquad  $A \cap B = A \cap C$, every $x \in A$ must also be contained in $C$ simultaneously.\\
\qquad  There is no such $x$ where $x \in A \cap B$ and $x \not \in A \cap C$ and vice versa.\\ 
\qquad  By rule of intersection, no such $x$ exists such that $x \in B$ and $x \not \in C$ and \\
\qquad vice versa. \\ 
\end{flushleft}

% STATEMENT 2
\begin{flushleft}
 (2)\:\: Assume for every $x \in A \cup B$ that $x \not \in A$. $x$ must therefore explicitly be \\ \qquad contained in $B$. Since $A \cup B = A \cup C$ and $x \not \in A$, x must also explicitly be \\\qquad contained in $C$. Therefore no such $x$ exists that $x \in B$ and $x \not \in C$ and vice \\ \qquad versa. \\
\end{flushleft}

% BEGINNING OF CONCLUSION
\begin{flushleft}
\textbf{CONSLUSION}\\
\end{flushleft}

\begin{flushleft}
\qquad Statement (1) showed that every $x \in A \cap B$ and $x \in A \cap C$ must mean \\ 
\qquad that every $x \in B$ and $x \in C$ at the same time. Additionally, statement (2) \\
\qquad showed that every $x \in A \cup B$ and $x \in A \cup C$ simultaneously. By \\
\qquad considering both statements to be true, there can not exist an $x \in B$ \\
\qquad where $x \not \in C$. Hence, the definition of subsets and the equality of sets $B$ \\
\qquad and $C$.
\end{flushleft}

\begin{flushleft}
Show that assuming only one of the conditions in the part I is not sufficient to prove $B = C$ \\
\end{flushleft}

\begin{flushleft}
 (1)\:\: Only assuming that $A \cap B = A \cap C$ is not sufficient because there could \\
 \qquad exist an $x \in B$ and  $x \not \in C$ because $A \cap B$ and $A \cap C$ are not said to be \\ \qquad equal.
\end{flushleft}

\vspace{2cm}

\begin{flushleft}
 (2)\:\: Only assuming that $A \cup B = A \cup C$ is not sufficient because there could \\ 
 \qquad exist an $x \in A \cap B$ and  $x \not \in A \cap C$. $A \cup B$ and $A \cup C$ do not elimate this \\ 
 \qquad possibility independently.
\end{flushleft}
%----------------------------------------------------------------------------------------

\end{document}Jacobs-MacBook-Pro:code_snippet jacobshiohira$ 
