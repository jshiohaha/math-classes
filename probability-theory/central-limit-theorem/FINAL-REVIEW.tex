\documentclass[12pt]{article}
\usepackage[margin=1in]{geometry} 
\usepackage{amsmath,amsthm,amssymb,amsfonts}
\usepackage{enumitem}
\usepackage{tabu}
\usepackage{xcolor}
 \usepackage{mathtools}
 \usepackage{gensymb}
 
\newcommand{\N}{\mathbb{N}}
\newcommand{\Z}{\mathbb{Z}}

\newcommand{\nspace}{\vspace*{.5cm}}
\newcommand{\nline}{\nspace \noindent}

\newcommand{\expected}[1]{\text{E}(#1)}

\newenvironment{nscenter}
 {\parskip=0pt\par\nopagebreak\centering}
 {\par\noindent\ignorespacesafterend}
 
\def\SPSB#1#2{\rlap{\textsuperscript{\textcolor{black}{#1}}}\SB{#2}}
 
% pg 338: 1,2,3,5,10,12
 
\begin{document}
\title{Math 487 Final Review}
\author{Jacob Shiohira}
\maketitle

\noindent
\textbf{Q1} Let $S_{100}$ be the number of heads that turn up in $100$ tosses of a fair coin. Use the Central Limit Theorem to estimate

\begin{enumerate}[label=(\alph*)]
\item $\mathbb{P}(S_{100} \leq 45)$

\begin{align*}
\mathbb{P}(S_{100} \leq 45) &= \mathbb{P}\Bigg ( \frac{S_{100}}{\sqrt{100}} \leq \frac{45-(100 \cdot .5)}{\sqrt{100 \cdot .5 \cdot .5)}} \Bigg ) \\
&= \mathbb{P} \Bigg ( \frac{S_{100}}{\sqrt{100}} \leq -1 \Bigg ) \\
&= .1587
\end{align*}

\item $\mathbb{P}(45 < S_{100} < 55)$

\begin{align*}
\mathbb{P}(S_{100} < 45) &= \mathbb{P}\Bigg (\frac{45.5-(100 \cdot .5)}{\sqrt{100 \cdot .5 \cdot .5)}} < \frac{S_{100}}{\sqrt{100}} < \frac{54.5-(100 \cdot .5)}{\sqrt{100 \cdot .5 \cdot .5)}} \Bigg ) \\
&= \mathbb{P} \Bigg (-.9 < \frac{S_{100}}{\sqrt{100}} < .9 \Bigg ) \\
&= \mathbb{P} \Bigg (\frac{S_{100}}{\sqrt{100}} < .9 \Bigg )  - \mathbb{P} \Bigg (-.9 < \frac{S_{100}}{\sqrt{100}} \Bigg ) \\
&= .815594 - .18406 \\
& = 0.6315
\end{align*}

% TODO: one off .1 off on (c) and (d)
\item $\mathbb{P}(S_{100} > 63)$

\begin{align*}
\mathbb{P}(S_{100} > 63) &= \mathbb{P}\Bigg ( \frac{S_{100}}{\sqrt{100}} > \frac{63.5 - (100 \cdot .5)}{\sqrt{100 \cdot .5 \cdot .5)}} \Bigg ) \\
&= 1- \mathbb{P}\Bigg ( \frac{S_{100}}{\sqrt{100}} > 2.7 \Bigg ) \\
&= 1- .99653 \\
&= 0.00347
\end{align*}

\item $\mathbb{P}(S_{100} < 57)$
\end{enumerate}

\begin{align*}
\mathbb{P}(S_{100} < 57) &= \mathbb{P}\Bigg ( \frac{S_{100}}{\sqrt{100}} < \frac{56.5 - (100 \cdot .5)}{\sqrt{100 \cdot .5 \cdot .5)}} \Bigg ) \\
&= \mathbb{P}\Bigg ( \frac{S_{100}}{\sqrt{100}} < 1.3 \Bigg ) \\
&= .9032 
\end{align*}

\noindent
\textbf{Q2} Let $S_{200}$ be the number of heads that turn up in $200$ tosses of a fair coin. Estimate

\begin{enumerate}[label=(\alph*)]
\item $\mathbb{P}(S_{200} = 100)$

\begin{align*}
\mathbb{P}(S_{200} = 100) &\sim \frac{\phi(X_{100})}{\sqrt{200 \cdot .5 \cdot .5}} = \frac{\phi(0)}{\sqrt{50}} = \frac{1}{\sqrt{50}} \Bigg ( \frac{1}{\sqrt{2 \pi}}\Bigg )\\
&= .056419
\end{align*}

\item $\mathbb{P}(S_{200} = 90)$
\begin{align*}
\mathbb{P}(S_{200} = 90) &\sim \frac{\phi(X_{90})}{\sqrt{200 \cdot .5 \cdot .5}} = \frac{\phi(- \sqrt{2})}{\sqrt{50}} = \frac{1}{\sqrt{50}} \Bigg ( \frac{1}{\sqrt{2 \pi}} e^{- \frac{(-\sqrt{2})^2}{2}} \Bigg )\\
&= \frac{1}{\sqrt{50}} \Bigg ( \frac{1}{\sqrt{2 \pi}} e^{-1} \Bigg ) \\
&= .020755
\end{align*}

\item $\mathbb{P}(S_{200} = 80)$

\begin{align*}
\mathbb{P}(S_{200} = 80) &\sim \frac{\phi(X_{80})}{\sqrt{200 \cdot .5 \cdot .5}} = \frac{\phi(- \sqrt{8})}{\sqrt{50}} = \frac{1}{\sqrt{50}} \Bigg ( \frac{1}{\sqrt{2 \pi}} e^{- \frac{(-\sqrt{8})^2}{2}} \Bigg )\\
&= \frac{1}{\sqrt{50}} \Bigg ( \frac{1}{\sqrt{2 \pi}} e^{-4} \Bigg ) \\
&= .001033
\end{align*}
\end{enumerate}

\noindent
\textbf{Q3} A true-false examination has $48$ questions. June has probability $3/4$ of answering a question correctly. April just guesses on each question. A passing score is $30$ or more correct answers. Compare the probability that June passes the exam with the probability that April passes it.

\vspace*{.3cm}
\noindent
Note that $\mathbb{P}(S_{48} \geq 30) = 1 - \mathbb{P}(S_{48} < 30) = 1 - \mathbb{P}(S_{48} \leq 29.5)$. We will use this fact to calculate the probability that June and April pass the exam.

\begin{align*}
\mathbb{P}(S_{48} \leq 29.5) &= \mathbb{P}\Bigg ( \frac{S_{48}}{\sqrt{48}} \leq \frac{29.5 - (48 \cdot .75)}{\sqrt{48 \cdot .75 \cdot .25)}} \Bigg ) \\
&= 1- \mathbb{P}\Bigg ( \frac{S_{48}}{\sqrt{48}} \leq -2.1667 \Bigg ) \\
&=  1- z(-2.1667) \\
&= 0.985
\end{align*}

\begin{align*}
\mathbb{P}(S_{48} \leq 29.5) &= \mathbb{P}\Bigg ( \frac{S_{48}}{\sqrt{48}} \leq \frac{29.5 - (48 \cdot .5)}{\sqrt{48 \cdot .5 \cdot .5)}} \Bigg ) \\
&= 1- \mathbb{P}\Bigg ( \frac{S_{48}}{\sqrt{48}} \leq 1.5877 \Bigg ) \\
&=  1- z(1.5877) \\
&= 0.05705
\end{align*}

\noindent
So, it can be seen that June has a $98.5\%$ chance of passing the exam while April only has a $5.7\%$ chance of passing the exam.

\vspace*{.3cm}
\noindent
\textbf{Q5} A rookie is brought to a baseball club on the assumption that he will have a $.300$ batting average. (Batting average is the ratio of the number of hits to the number of times at bat.) In the first year, he comes to bat $300$ times and his batting average is $.267$. Assume that his at bats can be considered Bernoulli trials with probability $.3$ for success. Could such a low average be considered just bad luck or should he be sent back to the minor leagues? Comment on the assumption of Bernoulli trials in this situation.

If his batting average was $.267$ out of $300$ hits, we know he had approximately $80$ hits. So, the probability we want to analyze is the probability that player is likely to have less than $80$ hits out of $300$ at bats with a success rate of $.3$.

\begin{align*}
\mathbb{P}(S_{300} \leq 80) &= \mathbb{P}\Bigg ( \frac{S_{300}}{\sqrt{300}} \leq \frac{80 - (300 \cdot .3)}{\sqrt{300 \cdot .3 \cdot .7)}} \Bigg ) \\
&= \mathbb{P}\Bigg ( \frac{S_{300}}{\sqrt{300}} \leq -1.26 \Bigg ) \\
&=  .10383
\end{align*}


\noindent
\textbf{Q10} Find the probability that among $10,000$ random digits the digit $3$ appears not more than $931$ times.

\begin{align*}
\mathbb{P}(S_{10,000} \leq 931) &= \mathbb{P}\Bigg ( \frac{S_{10,000}}{\sqrt{10,000}} \leq \frac{931 - (10,000 \cdot .1)}{\sqrt{10,000 \cdot .1 \cdot .9)}} \Bigg ) \\
&= \mathbb{P}\Bigg ( \frac{S_{10,000}}{\sqrt{10,000}} \leq -2.3 \Bigg ) \\
&=  .0107
\end{align*}


\noindent
\textbf{Q12} A balanced coin is flipped $400$ times. Determine the number $x$ such that the probability that the number of heads is between $200 - x$ and $200 + x$ is  approximately $.80$.

\begin{align*}
\mathbb{P}( 200 - x \leq S_{400} \leq 200 + x) &= \mathbb{P}\Bigg ( \frac{(200 - x) - (400 \cdot .5)}{\sqrt{400 \cdot .5 \cdot .5)}} \leq S_{400} \leq \frac{(200 + x) - (400 \cdot .5)}{\sqrt{400 \cdot .5 \cdot .5)}} \Bigg ) = .8 \\
&= \mathbb{P}\Bigg ( \frac{(200 - x) - 200}{10} \leq S_{400} \leq \frac{(200 + x) - 200}{10} \Bigg ) = .8 \\
&= \mathbb{P}\Bigg ( \frac{-x}{10} \leq S_{400} \leq \frac{x}{10} \Bigg ) = .8 \\
&= 2 \cdot \mathbb{P}\Bigg ( 0 \leq S_{400} \leq \frac{x}{10} \Bigg ) = .8 \\
&= \mathbb{P}\Bigg (S_{400} \leq \frac{x}{10} \Bigg ) - \mathbb{P}\Bigg (S_{400} \leq 0 \Bigg ) = .4 \\
&= \mathbb{P}\Bigg (S_{400} \leq \frac{x}{10} \Bigg ) = .9 \\
&= \mathbb{P}\Bigg (S_{400} \leq 1.28 \Bigg ) = .9 \\
\end{align*}

\noindent
So, we know that $\frac{x}{10} = 1.28$. Thus, $x \sim 13$.

\end{document} 