\documentclass[12pt]{article}
\usepackage[margin=1in]{geometry} 
\usepackage{amsmath,amsthm,amssymb,amsfonts}
\usepackage{enumitem}
\usepackage{tabu}
\usepackage{xcolor}
 \usepackage{mathtools}
 \usepackage{gensymb}
 
\newcommand{\N}{\mathbb{N}}
\newcommand{\Z}{\mathbb{Z}}

\newcommand{\nspace}{\vspace*{.5cm}}
\newcommand{\nline}{\nspace \noindent}

\newcommand{\expected}[1]{\text{E}(#1)}

\newenvironment{nscenter}
 {\parskip=0pt\par\nopagebreak\centering}
 {\par\noindent\ignorespacesafterend}
 
\def\SPSB#1#2{\rlap{\textsuperscript{\textcolor{black}{#1}}}\SB{#2}}
 
 % 6.1 4,17,18,21
 % 6.2 2,4,5,9,28,29
 
\begin{document}
\title{Math 487 Homework 5}
\author{Jacob Shiohira}
\maketitle

\subsection*{Section 6.1}

% In Las Vegas the roulette wheel has a $0$ and a $00$ and then the numbers $1$ to $36$ marked on equal slots; the wheel is spun and a ball stops randomly in one slot. When a player bets $1$ dollar on a number, he receives $36$ dollars if the ball stops on this number, for a net gain of $35$ dollars; otherwise, he loses his dollar bet. Find the expected value for his winnings.

\noindent
\textbf{Ch 6.1 Q4} With $38$ possible spots to place a bet on, we know that the probability of winning $35$ dollars is $1/38$, and the probability of losing a dollar is $37/38$. Let $X$ be the random variable which denotes your winnings on a $1$ dollar bet in Las Vegas roulette. Then, the distribution of $X$ is given by

\begin{equation*}
m_X = {-1 \qquad 35 \choose 37/38 \quad 1/38}.
\end{equation*}

\nline
Thus, we see then that $-1 \cdot (37/38) + 35 \cdot (1/38) = - 1/19$.


% Let $X$ be the first time that a failure occurs in an infinite sequence of Bernoulli trials with probability $p$ for success. Let $p_k = P(X = k)$ for $k = 1, 2, \ldots$. Show that $p_k = p^{k - 1}q$ where $q = 1 - p$. Show that $\sum_{k} p_k = 1$. Show that $\text{E}(X) = 1/q$. What is the expected number of tosses of a coin required to obtain the first tail?

\nline
\textbf{Ch 6.1 Q17} Since $X$ is the first time that a failure occurs in an infinite sequence, let's assume that the failure occurs after $k$ trials. Then, the probability of a failure is $k-1$ multiplications of $p$ and $1$ multiplication of $1-p$. This is valid because $k-1+1=k$ for our $k$ trials. Thus, for $q=1-p$, we can formally see that

\begin{equation*}
p_k = p^{k-1} (1-p) = p^{k-1} q.
\end{equation*}

\nline
Then, we can show $\sum_{k} p_k = 1$. Since $k=1,2, \ldots$, the limits of the summation are originally $k=1$ to $\infty$.

\begin{align*}
\sum_{k} p_k &= \sum_{k=1}^{\infty} p^{k-1} q \\
&= q \sum_{k=0}^{\infty} p^{k} \\
&= q \Big ( \frac{1}{1-p} \Big ) \\
&= q \frac{1}{q} \\
&= 1.
\end{align*}

\nline
In this case, $p_k = p^{k-1} q$ for $k = 1,2,\ldots$. So,

\begin{align*}
\text{E}(X) &= 1q + 2pq + 3p^2q + \ldots \\
&= q (1 + 2p + 3p^2 + \ldots).
\end{align*}

\nline
Then, we can see that the infinite series $1 + p + p^2 + \ldots$, which is equivalent to $\frac{1}{1-p}$, is embedded in the previous equation. By differentiating this, we get $1 + 2p + 3p^2 + \ldots$, which is the term we have above and is equivalent to $\frac{1}{(1-p)^2}$. By substituting into the expanded equation of $\text{E}(X)$, we get 

\begin{align*}
\text{E}(X) &= q (1 + 2p + 3p^2 + \ldots) \\
&= q \frac{1}{(1-p)^2} \\
&= q \Big ( \frac{1}{q^2} \Big ) \\
&= \frac{1}{q}.
\end{align*}

\nline
Thus, the expected value of the first tail of a fair coin is $\frac{1}{q} = \frac{1}{\frac{1}{2}} = 2$. 

%  Exactly one of six similar keys opens a certain door. If you try the keys, one after another, what is the expected number of keys that you will  have to try before success?

\nline
\textbf{Ch 6.1 Q18} To compute the expected value, $\text{E}(X)$ for some random variable $X$, we can use the definition of $\text{E}(X)$ using a sum and a distribution function,

\begin{align*}
\text{E}(X) &= \sum_{k=1}^{6} k \cdot \frac{1}{6} \\
&= \frac{1}{6} + \frac{2}{6} + \frac{3}{6} + \frac{4}{6} + \frac{5}{6} + \frac{6}{6} \\
&= \frac{7}{2}.
\end{align*}

% Let $X$ be a random variable which is Poisson distributed with parameter $\lambda$. Show that E$(X) = \lambda$. Hint: Recall that

% \begin{equation*}
% e^{x} = 1 + x + \frac{x^2}{2!} + \frac{x^3}{3!} + \cdots .
% \end{equation*}

\nline
\textbf{Ch 6.1 Q21} To compute the expected value of $X$, we need to note that the distribution function of $X$ is $p(x) = \frac{ e^{- \lambda} \lambda^x }{x!}$. Then, 

\begin{align*}
\text{E}(X) = \sum_{x=0}^{\infty} x \cdot p(x) &= \sum_{x=1}^{\infty} x \cdot  \frac{ e^{- \lambda} \lambda^x }{x!} \\
&= \sum_{x=1}^{\infty} \frac{ e^{- \lambda} \lambda^x }{(x-1)!} \\ 
&= e^{- \lambda} \sum_{x=1}^{\infty} \frac{ \lambda^x }{(x-1)!} \\ 
\end{align*}

\nline
Now, we see that the denominator yields $0$ when $x=1$, so we can set $k=x-1$ and adjust the summation, 

\begin{align*}
\text{E}(X) &= \lambda e^{- \lambda} \sum_{k=1}^{\infty} \frac{ \lambda^k }{k!} \\ 
\end{align*}

\nline
Now, we see that $\sum_{k=1}^{\infty} \frac{ \lambda^k }{k!} = 1 + x + \frac{\lambda^2}{2!} + \frac{\lambda^3}{3!} + \cdots$, which is equivalent to $e^{\lambda}$. By substituting into $\text{E}(X)$, we get 

\begin{equation*}
\text{E}(X) = \lambda e^{- \lambda} e^{\lambda} = \lambda.
\end{equation*}

\subsection*{Section 6.2}

\nline
\textbf{Ch 6.2 Q2} A random variable $X$ has the distribution

\begin{equation*}
p_x = {0 \qquad 1 \qquad 2 \qquad 4 \choose 1/3 \quad 1/3 \quad 1/6 \quad 1/6}
\end{equation*}

\nline
Find the expected value, variance, and standard deviation of $X$.

\begin{align*}
\text{E}(X) &= 0 \cdot \frac{1}{3} + 1 \cdot \frac{1}{3} + 2 \cdot \frac{1}{6} + 4 \cdot \frac{1}{6} \\
&= \frac{1}{3} + \frac{2}{6} + \frac{4}{6} \\
&= \frac{4}{3}.
\end{align*}

\begin{align*}
\text{V}(X) &= \text{E}(X^2) - \Big ( \frac{4}{3} \Big )^2 \\
&= \Big [ 0^2 \cdot \frac{1}{3} + 1^2 \cdot \frac{1}{3} + 2^2 \cdot \frac{1}{6} + 4^2 \cdot \frac{1}{6} \Big ] - \frac{16}{9} \\
&= \Big [ \frac{11}{3} \Big ] - \frac{16}{9} \\
&= \frac{17}{9}.
\end{align*}

\begin{align*}
\text{D}(X) = \sqrt{\text{V}(X)} = \sqrt{\frac{17}{9}} = \frac{\sqrt{17}}{3}.
\end{align*}

\nline
\textbf{Ch 6.2 Q4} $X$ is a random variable with $\text{E}(X) = 100$ and $\text{V}(X) = 15$. Find

\begin{enumerate}[label=(\alph*)]
\item $\expected{X^2}$

\nline
We can rearrange $\text{V}(X) = \text{E}(X^2) - \text{E}(X)^2$ for $\text{E}(X^2)$, which yields $\text{V}(X) + \text{E}(X)^2$. Plugging in values, we see that $\expected{X^2} = 10015$.

\item $\expected{3X + 10}$

\nline
We can use the fact that $\expected{cX} = c \expected{X}$. So, $\expected{3X + 10}$ is equivalent to $3 \expected{X} + 10$. Thus, $\expected{3X + 10} = 310$. 

\item $\expected{-X}$

\nline
We can use the fact that $\expected{cX} = c \expected{X}$. So, $\expected{-X}$ is equivalent to $-1 \expected{X}$. Thus, $\expected{-X} = -100$. 

\item $\text{V}(-X)$

\nline
We can use the fact that $V(cX) = c^2 V(X)$. So, $\text{V}(-X) = -1^2 \text{V}(X)$. Thus,  $\text{V}(-X) = 15$.

\item $\text{D}(-X)$

\nline
We can use the previous result that said $\text{V}(-X) = 15$ since $\text{D}(-X) = \sqrt{\text{V}(-X)}$. Thus, $\text{D}(-X) = \sqrt{15}$.

\end{enumerate}

\nline
\textbf{Ch 6.2 Q5} In a certain manufacturing process, the (Fahrenheit) temperature never varies by more than $2^{\circ}$ from $62^{\circ}$. The temperature is, in fact, a random variable $F$ with distribution

\begin{equation*}
P_F = { 60 \qquad 61 \qquad 62 \qquad  63 \qquad 64 \choose 1/10 \quad 2/10 \quad 4/10 \quad 2/10 \quad 1/10}.
\end{equation*}

\begin{enumerate}[label=(\alph*)]
\item Find $\expected{F}$ and $\text{V}(F)$.

\begin{equation*}
\expected{F} = 60 \cdot \frac{1}{10} + 61 \cdot \frac{2}{10} + 62 \cdot \frac{4}{10} + 63 \cdot \frac{2}{10} + 64 \cdot \frac{1}{10} = 62.
\end{equation*}

\begin{equation*}
\text{V}(F) = \Big [ 60^2 \cdot \frac{1}{10} + 61^2 \cdot \frac{2}{10} + 62^2 \cdot \frac{4}{10} + 63^2 \cdot \frac{2}{10} + 64^2 \cdot \frac{1}{10} \Big ] - 62^2 = 1.2.
\end{equation*}

\item Define $T = F - 62$. Find $\expected{T}$ and $\text{V}(T)$, and compare these answers with those in part $(a)$.

\nline
We know that $\expected{X + a} = \expected{X} + a$, so $\expected{T} = \expected{F} - 62$. Thus, $\expected{T} = 0$. Additionally, we know that $\text{V}(X + a) = \text{V}(X)$, so $\text{V}(T) = \text{V}(F)$. Thus, $\text{V}(T) = 1.2$.


\item It is decided to report the temperature readings on a Celsius scale, that
is, $C = (5/9) (F - 32)$. What is the expected value and variance for the
readings now? 

\nline
We know that $\expected{X + a} = \expected{X} + a$, so $\expected{C} = \frac{5}{9} \Big ( \expected{F} - 32 \Big )$. Thus, $\expected{C} = \frac{50}{3}$. Additionally, we know that $\text{V}(cX + a) = c^2 \text{V}(X)$, so $\text{V}(C) = \frac{5}{9}^2 \cdot \text{V}(F)$. Thus, $\text{V}(C) = \frac{5}{9}^2 \cdot \frac{6}{5}= \frac{10}{27}$.
\end{enumerate}

\nline
\textbf{Ch 6.2 Q9} A die is loaded so that the probability of a face coming up is proportional to the number on that face. The die is rolled with outcome $X$. Find $\text{V}(X)$ and $\text{D}(T)$.

\nline
The total of all faces on a die is $21$. Thus, the probability of rolling a $1$ is $1/21$, rolling a $2$ is $2/21$, and so on. Thus, we can compute expected value using the summation definition.

\begin{equation*}
\sum_{x=1}^{6} x \cdot \frac{x}{21} = \frac{13}{3}.
\end{equation*}

\nline
Then, $\text{V}(X)  = \expected{X^2} - \expected{X}^2$. So, 

\begin{align*}
\text{V}(X) &= \Big ( \sum_{x=1}^{6} x^2 \cdot \frac{x}{21} \Big ) - \frac{13}{3}^2 \\
&= 21 - \frac{13}{3}^2 = \frac{20}{9}.
\end{align*}

\nline
Lastly, $\text{D}(X) = \sqrt{\text{V}(X)} = \sqrt{ \frac{20}{9}} = \frac{\sqrt{20}}{3}$.

% TODO: finish this by getting help? 
\nline
\textbf{Ch 6.2 Q28} In Example $5.3$, assume that the book in question has $1000$ pages. Let $X$ be the number of pages with no mistakes. Show that $\expected{X} = 905$ and $\text{V}(X) = 86$. Using these results, show that the probability is $\leq .05$ that there will be more than $924$ pages without errors or fewer than $866$ pages without errors.

\nline
The likelihood that there are no mistakes on a page can is given by $\frac{ e^{- 0.1} 0.1^0 }{0!} = e^{-.1} = 0.9048374$. Since the likelihood of a mistake (or lack thereof) on a page is independent of other pages, we can multiply this probability by $1000$, which represents the number of pages in the book, to find the expected value of pages without mistakes. So, $\expected{X} = 1000 \cdot e^{-.1} = 905$ pages.

\nline
Per the usual definition, we know $\text{V}(X) = \expected{X^2} - \expected{X}^2$. However, since we know that a mistake on the $i^{\text{th}}$ page does not affect a mistake on the $i+1^{\text{th}}$ page, we can reduce the equation to 

\begin{align*}
\text{V}(X) &= \expected{X} - \expected{X}^2 \\
&= \Bigg [ \frac{ e^{- 0.1} 0.1^0 }{0!} - \Bigg ( \frac{ e^{- 0.1} 0.1^0 }{0!} \Bigg )^2 \Bigg ] \\
&= \frac{ e^{- 0.1} 0.1^0 }{0!} \Bigg ( 1- \frac{ e^{- 0.1} 0.1^0 }{0!} \Bigg ) \\
&= e^{- 0.1} \Bigg ( 1- e^{- 0.1} \Bigg ) \\
& \sim 0.086107 
\end{align*}

\nline
So, for total variance see that $\text{V}(X) = 1000 \cdot 0.086107 = 86$.

\nline
We can use the cumulative distribution function for a binomial distribution to show $\mathbb{P}(X > 924) \leq .05$ and $\mathbb{P}(X < 866) \leq .05$. We see that $F(924; 0.9048374, 1000) \sim 0.985$, so the probability of more than $924$ pages is approximately $.015$. Additionally, $F(866; 0.9048374, 1000) \sim 4.209053 \times 10^{-5}$, and this is obviously less than $.05$.

%http://www.chegg.com/homework-help/questions-and-answers/book-200-pages-number-mistakes-page-apoisson-random-variable-mean-001-independent-thenumbe-q228411
\nline
\textbf{Ch 6.2 Q29} Let $X$ be Poisson distributed with parameter $\lambda$. Show that $\text{V}(X) = \lambda$.   

\nline
To find $\text{V}(X)$ of the Poisson distributed random variable $X$ with parameter $\lambda$, we can use the equation

\begin{equation*}
\text{V}(X) = \expected{X^2} - \expected{X}^2.
\end{equation*}

\nline
To find $\expected{X^2}$, we need a second order value of $X$, but there is no $i,j$ where $i \neq j$ such that $X=x_i=x_j$. So, $\expected{X^2} = \expected{X(X-1) + X}$.

\begin{align*}
\expected{X(X-1) + X} &= \sum_{x=0}^{\infty} x(x-1) \frac{ e^{- \lambda} \lambda^x }{x!} \\
&= \sum_{x=0}^{\infty} \frac{ e^{- \lambda} \lambda^x }{(x-2)!} \\
&= \lambda^2  e^{- \lambda} \sum_{x=2}^{\infty} \frac{ \lambda^{x-2} }{(x-2)!} \\
&= \lambda^2  e^{- \lambda} e^{- \lambda} =  \lambda^2\\
\end{align*}

\nline
Thus, we can substitute values and we see that $\text{V}(X) = \lambda^2 + \lambda - \lambda^2 = \lambda$.
\end{document} 