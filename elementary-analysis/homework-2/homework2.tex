\documentclass{article}

%----------------------------------------------------------------------------------------

\usepackage{listings} % Required for inserting code snippets
\usepackage[usenames,dvipsnames]{color} % Required for specifying custom colors and referring to colors by name
\usepackage{amssymb}
\usepackage{amsmath}
\usepackage{mathtools}
\usepackage{tikz}

\definecolor{DarkGreen}{rgb}{0.0,0.4,0.0} % Comment color
\definecolor{highlight}{RGB}{255,251,204} % Code highlight color

%----------------------------------------------------------------------------------------

\begin{document}

%----------------------------------------------------------------------------------------

\begin{center}
Prove that (L) $(A \cap B) \cup C = A \cap (B \cup C)$ (R) if and only if $C \subset A$ 
\end{center}

\begin{flushleft}
\textbf{PROOF}
\end{flushleft}

% STATEMENT 1
\begin{flushleft}
 By definition, we know a set $C$ is a subset of another set $A$ iff all elements of $C$ are contained in $A$. In other words, there cannot be a single element $x$ that belongs to $C$ but does not belong to $A$. \\
 \vspace{.2cm}
 Suppose $x \in C$. We don't have to worry about the cases when $x$ doesn't belong to $C$ because our hypothesis assumes $C \subset A$.\\
  \vspace{.2cm}
 \textbf{(Case 1)} if $x \in B$ and $x \in A$ \\
  \vspace{.2cm}
 \textbf{(Case 2)} if $x \in B$ and $x \not \in A$\\ 
  \vspace{.2cm}
 \textbf{(Case 3)} if $x \not \in B$ and $x \in A$\\ 
  \vspace{.2cm}
 \textbf{(Case 4)} if $x \not \in B$ and $x \not \in A$\\ 
\end{flushleft}

% STATEMENT 1
\begin{flushleft}
 (1)\:\: If an element $x$ belongs to $A,B,C$ then $L=R$ regardless of the \\
 \qquad combination of intersections and unions. \\
 \end{flushleft}

% STATEMENT 2
\begin{flushleft}
 (2)\:\: We can evaluate at both $L$ and $R$ and then compare them. For $L$, an \\
 \qquad element $x$ does not belong to $A$ and $B$. However, it does belong to $C$, so \\
 \qquad $x$ exists in $L$. For $R$, an element $x$ belongs to $B$ or $C$. However, it does \\
 \qquad not belong to $A$. Since it is not in both, $x$ does not exist in $R$. $L \neq R$.\\
\end{flushleft}

% STATEMENT 3
\begin{flushleft}
 (3)\:\: Again, we can evaluate at both $L$ and $R$ and then compare them. For $L$, \\
 \qquad an element $x$ does not belong to $A$ and $B$. However, it does belong to $C$, \\
 \qquad so $x$ exists in $L$. For $R$, an element $x$ belongs to $B$ or $C$. Additionally, it  \\
 \qquad belongs to $A$ as well. Since it is in both, $x$ exists in $R$. $L=R$.\\
 \end{flushleft}


% STATEMENT 4
\begin{flushleft}
 (4)\:\: Again, we can evaluate at both $L$ and $R$ and then compare them. For $L$, \\
 \qquad an element $x$ does not belong to $A$ and $B$. However, it does belong to $C$, \\
 \qquad so $x$ exists in $L$. For $R$, an element $x$ belongs to $B$ or $C$. However, it does \\
 \qquad not belong to $A$. Since it is not in both, $x$ does not exist in $R$. $L \neq R$.\\
 \end{flushleft}

% BEGINNING OF CONCLUSION
\begin{flushleft}
\textbf{CONSLUSION}\\
\end{flushleft}

\begin{flushleft}
From \textbf{Cases 1-4}, we saw that the only time when the hypothesis held \\
true was \textbf{Case 1} and \textbf{Case 2}. In both instances, an element $x$ belonged \\
to both $C$ and $A$. The value of the equation never depended on whether \\
or not $x$ belonged to $B$. So, by definition, $C \subset A$ because when $x$ belongs \\
to $C$, $x$ also belongs to $A$. \\
\end{flushleft}

%----------------------------------------------------------------------------------------

\end{document}
