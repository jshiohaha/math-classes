\documentclass[12pt]{article}
\usepackage[margin=1in]{geometry} 
\usepackage{amsmath,amsthm,amssymb,amsfonts}
\usepackage{enumitem}
\usepackage{tabu}
\usepackage{xcolor}
\usepackage{mathtools}
\usepackage{tcolorbox} 
\usepackage{changepage} 
\usepackage{kpfonts}
\usepackage{picture}
\usepackage{venndiagram}

\newcommand{\N}{\mathbb{N}}
\newcommand{\Z}{\mathbb{Z}}
\newcommand{\R}{\mathbb{R}}

\begin{document}
\section*{CHAPTER 7}

\noindent
\textbf{Problem 3:} Consider a card game, using the standard $52$-card deck, in which each of four players is dealt thirteen cards. Compute the probabilities that a specific player: $(a)$ has no clubs; $(b)$ has exactly ten clubs; $(c)$ has at least three of four aces.

\vspace*{5cm}

\noindent
\textbf{Problem 4:} (Cont.) In the same card game, what is the probability that each player is dealt a jack?

\vspace*{5cm}

\noindent
\textbf{Problem 9:} Recall that there are U.S. senators (two from each state).

\begin{enumerate}[label=(\roman*)]
\item If two senators are chosen at random, what is the probability that they are from the same state?
\item If the $100$ senators are organized into disjoint sets of two, what is the probability that, in each set, the two senators are from the same state?
\item In a committee of ten senators, what is the probability that no two are from the same state? (Assume that there are no restrictions on committee memberships)
\end{enumerate}

\vspace*{3cm}

% ================================================================================================================================
% ================================================================================================================================
\section*{CHAPTER 8}

\noindent
\textbf{Problem 3:} Compute the probability that, in a sample of $10$ people in the population at large, $2$ have their birthdays in May or June, $3$ have their birthdays in December or January, and the $5$  remaining ones have their birthdays during the rest of the year. (Just give the formula. Also, you may assume for simplicity that all the months have the same number of days.)

\vspace*{5cm}

\noindent
\textbf{Problem 4:} Two fair dice are thrown repeatedly until for the first time their sum exceeds $4$. What is the distribution of the trial number of that event?

\vspace*{5cm}

\noindent
\textbf{Problem 5:} Check the accuracy of the approximation of the binomial distribution with parameters $50$ and $\frac{1}{50}$ by the Poisson distribution with parameter $1$. Compute both distributions for the values $0$, $1$, $3$, and $5$.

\vspace*{5cm}

\noindent
\textbf{Problem 8:} In a particular ESP (extrasensory perception) experiment, an experimenter looks at one of five cards on each trial, and the subject is to guess at which card the experimenter is looking. Assuming that the subject does not have ESP, what is the distribution of the outcome (successful guess, unsuccessful guess) of a trial?

\vspace*{5cm}

% ================================================================================================================================
% ================================================================================================================================
\section*{CHAPTER 9}

\noindent
\textbf{Problem 1:} Archie has three coins in his pocket: a standard coin, a coin with heads on both sides, and a coin with tails on both sides. He pulls one coin out of his pocket, looks at one side of the coin, and notices that it is a tail. He reasons that the probability of seeing a head on the other side of this coin is $\frac{1}{2}$. Do you agree with his reasoning?

\vspace*{5cm}

\noindent
\textbf{Problem 2:} Let $a$, $b$, and $c$ be outcomes in some finite sample space $\Omega$ having $2^{\Omega}$ as a field of events, with some probability measure $\mathbb{P}$. You are told that $\mathbb{P}(\{ a,b \} \lvert \{ b,c \}) = \alpha$ and that $\mathbb{P}(\{ c \}) = \beta$.

\begin{enumerate}[label=(\roman*)]
\item Compute $\mathbb{P}(\{ b \})$ in terms of $\alpha$ and $\beta$.
\item Give some possible values for $\alpha$ and $\beta$.
\item Find constraints on $\alpha$ and $\beta$, that is, find a general expression constraining the possible values of $\alpha$ and $\beta$.
\item Find an expression constraining the possible values of $\mathbb{P}(\{ a \})$.
\end{enumerate}

\vspace*{5cm}

\noindent
\textbf{Problem 8:} Consider, in some ethnic group, the set of all families having two children. Let us assume that, for such families, the probability of having a boy is $.5$. Take one family at random and suppose that one of their children is a boy. What is the probability that the other child is also a boy?

\vspace*{5cm}

\noindent
\textbf{Problem 12:} An astronomer has detected punctual signals from an unknown source in the sky. The signals are of two kinds, which she denotes `$A$' and `$B$.' She assumes that the occurrence of the signals is governed by a random process, namely, that the number of signals of any kind received in the course of one hour has a Poisson distribution with parameter $\lambda$. When a signal occurs, it is an `$A$' signal with probability $\theta$ and a `$B$' signal with probability $1-\theta$. Write a formula for the distribution of `$A$'  signals received in one hour.

\vspace*{5cm}

\end{document} 